\documentclass{article}
\topmargin = 0in
\oddsidemargin = 0in
\evensidemargin = \oddsidemargin
\textwidth = 6.5in
\textheight = 8in
\usepackage{times}
\usepackage{bcprules}
\usepackage{amsthm}
\usepackage{syntax}
\usepackage{trfrac}
\usepackage{mathtools}

\renewcommand{\syntleft}{\normalfont\itshape}
\renewcommand{\syntright}{}

\newcommand{\step}[2]{{\tt #1} $\longrightarrow$ {\tt #2}}
\newcommand{\eval}[2]{{\tt #1} $\Downarrow$ {\tt #2}}
\newcommand{\tc}[3]{{\tt #1} $\vdash$ {\tt #2} \ : \ {\tt #3}}
\newcommand{\tcDef}[2]{{\tt #1}\ : \ {\tt #2}}

\newcommand{\inferrule}[3]{\infrule[#1]{\mbox{#2}}{\mbox{#3}}}
\newcommand{\inferax}[2]{\infrule[#1]{\mbox{}}{\mbox{#2}}}


\title{Homework 2}

\author{John Bender and Lorenzo Gomez, CS 231}

\date{October 17th, 2013}

\newtheorem{free}{Free Variables}
\newtheorem{deter}{Determinism}
\newtheorem{thm}{Theorem}

\begin{document}

\maketitle

\begin{description}
\item A description is needed here.
  \begin{enumerate}
  \item Call-by-value Lambda Calculus
    \begin{enumerate}
      \item $x\ (\lambda x.x)$ or $(\lambda y.y)\ z$
      \item $((\lambda x.x)\ \lambda y.y)\ r)\ (\lambda q.q)$ $\longrightarrow$ $((\lambda y.y)\ r)\ (\lambda q.q)$ $\longrightarrow$ $r\ (\lambda q.q)$ \\
      \item $((\lambda x. x)\ \lambda y.y)\ r)\ (\lambda q. q)$ \\
        \begin{equation*}
          \trfrac[app-left] {
          \trfrac[app-left]{
            \trfrac[app-left]{
              \trfrac[beta]{}{
                (\lambda x. x)\ s \longrightarrow s
              }
            }{
              (\lambda x. x)\ s)\ r \longrightarrow s\ r
            }
          }{
            ((\lambda x. x)\ s)\ r)\ q \longrightarrow (s\ r)\ q
          }
          }{
            (((\lambda x. x)\ s)\ r)\ q)\ p \longrightarrow
            ((s\ r)\ q)\ p
          }
        \end{equation*}

      \item $(\lambda x.x\ (\lambda y.y\ \lambda z.z))\ (\lambda w. w \ \lambda r.r)$ \\
        \begin{equation*}
          \trfrac[E-App1]
          {
            \trfrac[E-App2]
            {
              \trfrac[E-AppBeta]
              {
                $ $ %empty premise
              }
              {
                \lambda y.y\ \lambda z.z  \longrightarrow \lambda z.z
              }
            }
            {
              \lambda x.x\ (\lambda y.y\ \lambda z.z) \longrightarrow \lambda x.x\ \lambda z.z
            }
          }
          {
            (\lambda x.x\ (\lambda y.y\ \lambda z.z))\ (\lambda w. w \ \lambda r.r) \longrightarrow
            (\lambda x.x\ \lambda z. z)\ (\lambda w.w\ \lambda r.r)
          }
        \end{equation*}



    \end{enumerate}

  \item{Stepping with call-by-value}
    \begin{enumerate}
    \item $(\lambda x.x\ x)\ ((\lambda y.y\ y)\ (\lambda z.z))$ $\longrightarrow$ \\
          $(\lambda x.x\ x)\ ((\lambda z.z)\ (\lambda z.z))$ $\longrightarrow$ \\
          $(\lambda x.x\ x)\ (\lambda z.z)$ $\longrightarrow$ \\
          $(\lambda z.z)\ (\lambda z.z)$ $\longrightarrow$ \\
          $(\lambda z.z)$

    \item $(\lambda x.\ (x\ (\lambda z.x\ z)))\ (\lambda x.x\ y\ x)$ $\longrightarrow$ \\
          $(\lambda x.x\ y\ x)\ (\lambda z.(\lambda x.x\ y\ x)\ z)\ $
$\longrightarrow$  \\
          $(\lambda z.(\lambda x.x\ y\ x)\ z)\ y\ (\lambda z.(\lambda x.x\ y\ x)\ z)$
$\longrightarrow$  \\
          $((\lambda x.x\ y\ x)\ y)\ (\lambda z.(\lambda x.x\ y\ x)\ z)$
$\longrightarrow$  \\
          $((y\ y\ y)\ (\lambda z.(\lambda x.x\ y\ x)\ z)$

          \newpage

    \item $(\lambda x.x\ x)\ (\lambda x.x\ x\ x)$ $\longrightarrow$ \\
          $(\lambda x.x\ x\ x)\ (\lambda x.x\ x\ x)$ $\longrightarrow$ \\
          $(\lambda x.x\ x\ x)\ (\lambda x.x\ x\ x)\ (\lambda x.x\ x\ x)$
$\longrightarrow$ \\
          $(\lambda x.x\ x\ x)\ (\lambda x.x\ x\ x)\ (\lambda x.x\ x\ x)\ (\lambda x.x\ x\ x)$ $\longrightarrow$ diverges


    \end{enumerate}

  \item{See attached $hw2a.ml$ file.}

  \item{Free Variables Proof}
    \begin{free}
      If FV(\verb|t|) = $\emptyset$ then either \verb|t| is a value or there exists at term \verb|t|$'$ such that \verb|t| $\longrightarrow$ \verb|t|$'$.
    \end{free}

    \textit{Induction Hypothesis.} If FV(\verb|t|$_0$) = $\emptyset$ and \verb|t|$_0$ is a sub-term of \verb|t|, then either \verb|t|$_0$ is a value or there exists at term \verb|t|$'_0$ such that \verb|t|$_0$ $\longrightarrow$ \verb|t|$'_0$.


    \begin{proof}
      We proceed by induction on \verb|t|:
      \begin{enumerate}
        \item If \verb|t| is a variable $x$, then FV(\verb|t|) is $\{x\}$, a contradiction.
        \item If \verb|t| is an abstraction, $\lambda x.$\verb|t|$_1$ then \verb|t| is a value.
        \item if \verb|t| is an application, $($\verb|t|$_1$ \verb|t|$_2)$, then \verb|t| is not a value and can take a step in one of three ways:
          \begin{enumerate}
            \item E-App1: By the inductive hypothesis \verb|t|$_1$ is either a value or
can take a step, in the latter case E-App1 applies.
            \item E-App2: By the inductive hypothesis \verb|t|$_2$ is either a value or can take a step, the latter case E-App2 applies.
            \item E-AppBeta: Otherwise, by the inductive hypothesis \verb|t|$_1$ and \verb|t|$_2$ are values and E-AppBeta applies.
          \end{enumerate}
      \end{enumerate}
    \end{proof}

  \item{Proof that $\omega$ Steps to Itself}
    \begin{thm}If $(\lambda x.x\ x)\ (\lambda x.x\ x)$ $\longrightarrow*$ $t$,
      then $t$ = $(\lambda x.x\ x)\ (\lambda x.x\ x)$.
    \end{thm}

    \textit{Induction Hypothesis.} If \verb|t|$_0$ $\longrightarrow$ \verb|t|$_0'$ , \verb|t|$_0$ $\longrightarrow$ \verb|t|$_0''$, and \verb|t|$_0$ $\longrightarrow$ \verb|t|$_0'$ is a sub-derivation of \verb|t| $\longrightarrow$ \verb|t|$'$, then \verb|t|$_0'$ = \verb|t|$_0''$.

    \begin{proof}
      The last rule in derivation of $t\ \longrightarrow*\ t$ can be 3 cases:
      \begin{enumerate}
        \item E-Refl: Then it is trivally true.

        \item E-Step:
            The last rule in derivation of $t\ \longrightarrow\ t$ can be 3 cases:
            \begin{enumerate}
            \item E-App1: Then we know $t$ is of form $(t1\ t2)$ and $t_1\ \longrightarrow\ t_1'$.
            \item E-App2: Then we know $t$ is of form $(v1\ t2)$ and $t_2\ \longrightarrow\ t_2'$.
            \item E-AppBeta: Then we know $t$ is of form $(t1\ v2)$.

            \verb|t|$_1$ must take a step to some \verb|t|$_1''$. By the inductive hypothesis, since we know \verb|t|$_1$ steps to both \verb|t|$_1'$ and \verb|t|$_1''$,
            then \verb|t|$_1'$ and \verb|t|$_1''$ must be the same.
            \end{enumerate}


        \item E-Trans: \verb|t|$_1$ must take a step to some \verb|t|$_1'$, and \verb|t| has the form
        \verb|if t|$_1$ \verb|then t|$_2$ \verb|else t|$_3$.


      \end{enumerate}
    \end{proof}

  \item[7]{Addition of Stuck Terms to BNF}

    \begin{grammar}
      <s> :=  <b>  \lit*{+} <n> | <n> \lit*{+} <b> | <b> \lit*{+} <b>
      \alt <b>  \lit*{>} <n> | <n> \lit*{>} <b> | <b> \lit*{>} <b>
      \alt \lit*{if} <n> \lit*{then} <t> \lit*{else} <t>

      <b> := \lit*{true} | \lit*{false}

      <n> := \ldots | \lit*{-1} | \lit*{0} | \lit*{1} | \ldots

      <t> := <t> \lit*{+} <t> | <t> \lit*{>} <t> | \lit*{if} <t> \lit*{then} <t> \lit*{else} <t>
    \end{grammar}

  \item[8]{Stuck Terms, Old vs. New Semantics}
    \begin{description}
      \item[(a)] \verb|if 1 then (1 > 2) else false|
      \item[(b)] \verb|if true then false else (true > 1)|
    \end{description}

  \item[9]{Eventually Stuck Terms, Old vs. New Semantics}
    \begin{description}
      \item[(a)] None. In the original semantics eventually stuck terms can exist at \verb|t|$_1$ and either \verb|t|$_2$ or \verb|t|$_3$ in an \verb|if|, but not both. In the modified semantics stuck terms can arise for all three sub-terms of an \verb|if|. So, the set of eventually stuck terms from the modified semantics forms a proper superset of those from the original semantics.
      \item[(b)] \verb|if true then (1 > 0) else (true > 1)|
    \end{description}

  \end{enumerate}

\end{description}

\end{document}
