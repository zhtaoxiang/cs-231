\documentclass{article}
\topmargin = 0in
\oddsidemargin = 0in
\evensidemargin = \oddsidemargin
\textwidth = 6.5in
\textheight = 8in
\usepackage{times}
\usepackage{bcprules}
\usepackage{amsthm}
\usepackage{syntax}
\usepackage{trfrac}
\usepackage{mathtools}

\renewcommand{\syntleft}{\normalfont\itshape}
\renewcommand{\syntright}{}

\newcommand{\step}[2]{{\tt #1} $\longrightarrow$ {\tt #2}}
\newcommand{\eval}[2]{{\tt #1} $\Downarrow$ {\tt #2}}
\newcommand{\tc}[3]{{\tt #1} $\vdash$ {\tt #2} \ : \ {\tt #3}}
\newcommand{\tcDef}[2]{{\tt #1}\ : \ {\tt #2}}

\newcommand{\inferrule}[3]{\infrule[#1]{\mbox{#2}}{\mbox{#3}}}
\newcommand{\inferax}[2]{\infrule[#1]{\mbox{}}{\mbox{#2}}}


\title{Homework 2}

\author{John Bender and Lorenzo Gomez, CS 231}

\date{October 17th, 2013}

\newtheorem{free}{Free Variables}
\newtheorem{omg}{Omega}

\begin{document}

\maketitle

\begin{enumerate}
  \item Call-by-value Lambda Calculus
    \begin{enumerate}
    \item $x\ (\lambda x.x)$
    \item $((\lambda x.x)\ \lambda y.y)\ r)\ (\lambda q.q)$ $\longrightarrow$ $((\lambda y.y)\ r)\ (\lambda q.q)$ $\longrightarrow$ $r\ (\lambda q.q)$ \\
    \item $(\lambda x.x\ (\lambda y.y\ (\lambda z.z\ \lambda n.n)))\ (\lambda w. w \ \lambda r.r)$ \\
      \begin{equation*}
        \trfrac[E-App1]{
          \trfrac[E-App2]{
            \trfrac[E-App2]{
              \trfrac[E-AppBeta]{}{
                \lambda z.z\ \lambda n.n \longrightarrow \lambda n.n
              }
            }{
              \lambda y.y\ (\lambda z.z\ \lambda n.n) \longrightarrow \lambda y.y\ \lambda n.n
            }
          }{
            \lambda x.x\ (\lambda y.y\ (\lambda z.z\ \lambda n.n)) \longrightarrow \lambda x.x\ (\lambda y.y\ \lambda n.n)
          }
        }{
          (\lambda x.x\ (\lambda y.y\ (\lambda z.z\ \lambda n.n)))\ \lambda w.w \longrightarrow
          (\lambda x.x\ (\lambda y.y\ \lambda n.n))\ \lambda w.w
        }
      \end{equation*}
    \end{enumerate}

    \vspace{0.5cm}

  \item{Stepping with call-by-value}
    \begin{enumerate}
    \item $(\lambda x.x\ x)\ ((\lambda y.y\ y)\ (\lambda z.z))$ $\longrightarrow$ \\
      $(\lambda x.x\ x)\ ((\lambda z.z)\ (\lambda z.z))$ $\longrightarrow$ \\
      $(\lambda x.x\ x)\ (\lambda z.z)$ $\longrightarrow$ \\
      $(\lambda z.z)\ (\lambda z.z)$ $\longrightarrow$ \\
      $(\lambda z.z)$

    \item $(\lambda x.\ (x\ (\lambda z.x\ z)))\ (\lambda x.x\ y\ x)$ $\longrightarrow$ \\
      $(\lambda x.x\ y\ x)\ (\lambda z.(\lambda x.x\ y\ x)\ z)\ $
      $\longrightarrow$  \\
      $(\lambda z.(\lambda x.x\ y\ x)\ z)\ y\ (\lambda z.(\lambda x.x\ y\ x)\ z)$
      $\longrightarrow$  \\
      $((\lambda x.x\ y\ x)\ y)\ (\lambda z.(\lambda x.x\ y\ x)\ z)$
      $\longrightarrow$  \\
      $(y\ y\ y)\ (\lambda z.(\lambda x.x\ y\ x)\ z)$

    \item $(\lambda x.x\ x)\ (\lambda x.x\ x\ x)$ $\longrightarrow$ \\
      $(\lambda x.x\ x\ x)\ (\lambda x.x\ x\ x)$ $\longrightarrow$ \\
      $(\lambda x.x\ x\ x)\ (\lambda x.x\ x\ x)\ (\lambda x.x\ x\ x)$
      $\longrightarrow$ \\
      $(\lambda x.x\ x\ x)\ (\lambda x.x\ x\ x)\ (\lambda x.x\ x\ x)\ (\lambda x.x\ x\ x)$ $\longrightarrow$ \bot


    \end{enumerate}

  \item{See attached $hw2a.ml$ file.}

  \item{Free Variables Proof}
    \begin{free}
      If FV(\verb|t|) = $\emptyset$ then either \verb|t| is a value or there exists at term \verb|t|$'$ such that \verb|t| $\longrightarrow$ \verb|t|$'$.
    \end{free}

    \textit{Induction Hypothesis.} If FV(\verb|t|$_0$) = $\emptyset$ and \verb|t|$_0$ is a sub-term of \verb|t|, then either \verb|t|$_0$ is a value or there exists at term \verb|t|$'_0$ such that \verb|t|$_0$ $\longrightarrow$ \verb|t|$'_0$.


    \begin{proof}
      We proceed by induction on the structure of \verb|t|:
      \begin{enumerate}
      \item If \verb|t| is a variable, then FV(\verb|t|) is not $\emptyset$, a contradiction.
      \item If \verb|t| is an abstraction, $\lambda x.$\verb|t|$_1$ then \verb|t| is a value.
      \item if \verb|t| is an application, $($\verb|t|$_1$ \verb|t|$_2)$, then \verb|t| is not a value. We proceed by induction on the structure of \verb|t|$_1$.
        \begin{enumerate}
        \item If \verb|t|$_1$ is a variable then FV(\verb|t|) is not $\emptyset$, a contradiction.
        \item If \verb|t|$_1$ is a function then by the inductive hypothesis \verb|t|$_2$ is either a value and E-AppBeta applies or \verb|t|$_2$ can take a step and E-App2 applies.
        \item If \verb|t|$_1$ is an application E-App1 applies.
        \end{enumerate}
      \end{enumerate}
    \end{proof}

  \item{Proof that $\omega$ Steps to Itself}
    \begin{omg} If $(\lambda x.x\ x)\ (\lambda x.x\ x)$ $\longrightarrow*$ \verb|t|, then \verb|t| = $(\lambda x.x\ x)\ (\lambda x.x\ x)$.
    \end{omg}

    \textit{Induction Hypothesis.} If $(\lambda x.x\ x)\ (\lambda x.x\ x)$ $\longrightarrow*$ \verb|t|$_0$ and $(\lambda x.x\ x)\ (\lambda x.x\ x)$ $\longrightarrow*$ \verb|t|$_0$ is a sub-derivation of $(\lambda x.x\ x)\ (\lambda x.x\ x)$ $\longrightarrow*$ \verb|t| then \verb|t|$_0$ = $(\lambda x.x\ x)\ (\lambda x.x\ x)$.

    \begin{proof}
      The last rule in derivation of $(\lambda x.x\ x)\ (\lambda x.x\ x)$ $\longrightarrow*$ $t$ can be 3 cases:
      \begin{enumerate}
      \item E-Refl: Then it is trivally true.

      \item E-Step:
        The last rule in derivation of \verb|t| $\longrightarrow$ \verb|t|$'$ can be 3 cases:
        \begin{enumerate}
          \item E-App1: Then we know \verb|t| is of form (\verb|t|$_1$\ \verb|t|$_2$) and \verb|t|$_1$ $\longrightarrow$ \verb|t|$_1'$, a contradiction.
          \item E-App2: Then we know \verb|t| is of form (\verb|v|$_1$\ \verb|t|$_2$) and \verb|t|$_2$ $\longrightarrow$ \verb|t|$_2'$, also a contradiction.
          \item E-AppBeta: Then we know \verb|t| is of form ($\lambda x.$\verb|t|$_0$) \verb|v|, where \verb|t|$_0$ is $x\ x$ and \verb|v| is $(\lambda x.x\ x)$. We substitute \verb|v| for $x$ in \verb|t|$_0$ and get $(\lambda x.x\ x)\ (\lambda x.x\ x)$, the desired result.
        \end{enumerate}

      \item E-Trans: Then \verb|t| $\longrightarrow *$ \verb|t|$''$ and \verb|t|$''$ $\longrightarrow *$ \verb|t|$'$. By the inductive hypothesis \verb|t|$''$ is $(\lambda x.x\ x)\ (\lambda x.x\ x)$, but then it also applies to \verb|t|$''$ $\longrightarrow *$ \verb|t|$'$ which means \verb|t|$'$ is $(\lambda x.x\ x)\ (\lambda x.x\ x)$, the desired result.

      \end{enumerate}
    \end{proof}

  \item Short circuit \verb|and| and \verb|or|.\\

    $newAnd = \lambda a.\lambda thnk.(a\ (\lambda x.\lambda y.x\ y)\ (\lambda x.\lambda y.y))\ thnk\ false$\\

    When $a$ is $true$ it will return $(\lambda x.\lambda y.x\ y)$ that will take the $thnk$ and apply it to $false$ ``unwrapping'' $thnk$'s contents. If $a$ is $false$ it will return $(\lambda x.\lambda y.y)$ that will ignore $thnk$ and return $false$: \\

    Similar to the above: \\

    $newOr = \lambda a.\lambda thnk.(a\ (\lambda x.\lambda y.y)\ (\lambda x.\lambda y.x\ y))\ thnk\ true$\\

  \item{See attached $hw2b.ml$ file.}
\end{enumerate}

\end{document}
