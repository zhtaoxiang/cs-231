\documentclass[12pt,letterpaper]{article}

%%%%%%%%%%%%%%%%%%%%%%%%%%%%%%%%%%%%%%%%%%%%%%%%%%%%%%%%%%%%%%%%%%%%%%%%%%%%%%%%
%%% Packages

\usepackage[margin=1in]{geometry}
\usepackage{lmodern}
\usepackage[T1]{fontenc}
\usepackage{bcprules}
\usepackage{amsbsy,amsfonts,amsmath,amssymb,amsxtra,latexsym,mathtools}
\usepackage{proof}
\usepackage{trfrac}

%%%%%%%%%%%%%%%%%%%%%%%%%%%%%%%%%%%%%%%%%%%%%%%%%%%%%%%%%%%%%%%%%%%%%%%%%%%%%%%%
%%% Title page

\title{COM SCI 231 Homework \#5}
\author{John Bender and Eric Marcin-Cuddy}
\date{due December 3, 2013}

%%%%%%%%%%%%%%%%%%%%%%%%%%%%%%%%%%%%%%%%%%%%%%%%%%%%%%%%%%%%%%%%%%%%%%%%%%%%%%%%
%%% Begin

\begin{document}

\maketitle

\begin{enumerate}

  % PROBLEM 1
  \item Refs
    \begin{enumerate}
      \item \texttt{(\ldots ) = (r := 42)}

      \item \texttt{(\ldots ) = (r := 42; ref 0)}

      \item \texttt{(\ldots ) = (let r = ref 5 in ($\lambda$x.(r := (!r + 1); !r)))}
    \end{enumerate}

 \item Proofs
   \begin{enumerate}

  % PROBLEM 2a
  \item \textbf{Theorem} \quad If $\varnothing \,|\, \Sigma \vdash t : T$ and $\varnothing \,|\, \Sigma \vdash \mu$, then either $t$ is a value or there exists a term $t^\prime$ and a store $\mu^\prime$ such that $t \,|\, \mu \to t^\prime \,|\, \mu^\prime$.\\
    \textbf{Proof} \quad By induction on the derivation of $\varnothing \,|\, \Sigma \vdash t : T$. Case analysis of the last rule used in the derivation.
      \begin{itemize}
      \item Case \textsc{T-Deref}: Then $t = \;!t_1$, $\varnothing \,|\, \Sigma \vdash t_1 : \text{Ref } T_1$, and $T=T_1$. By the induction hypothesis, either $t_1$ is a value or there exists a term $t_1^\prime$ and a store $\mu_1^\prime$ such that $t_1 \,|\, \mu \to t_1^\prime \,|\, \mu_1^\prime$.
        \begin{itemize}
        \item Case $t_1 \,|\, \mu \to t_1^\prime \,|\, \mu_1^\prime$: Then $t$ steps by \textsc{E-Deref}.
        \item Case $t_1$ is a value: By the Canonical Forms (Ref) Lemma, $t_1$ is a location $\ell$. By \textsc{T-Loc}, $\ell \in \text{dom}(\Sigma)$. Since $\varnothing \,|\, \Sigma \vdash \mu$, $\ell \in \text{dom}(\mu)$. Then $t$ steps by \textsc{E-DerefLoc}.
        \end{itemize}
      \end{itemize}

  % PROBLEM 2b
  \item \textbf{Theorem} \quad If $\varnothing \,|\, \Sigma \vdash t : T$, $\varnothing \,|\, \Sigma \vdash \mu$, and $t \,|\, \mu \to t^\prime \,|\, \mu^\prime$, then there exists a $\Sigma^\prime$ such that $\Sigma^\prime \supseteq \Sigma$, $\varnothing \,|\, \Sigma^\prime \vdash t^\prime : T$, and $\varnothing \,|\, \Sigma^\prime \vdash \mu^\prime$.\\
    \textbf{Proof} \quad By induction on the derivation of $\varnothing \,|\, \Sigma \vdash t : T$. Case analysis of the last rule used in the derivation.
      \begin{itemize}
      \item Case \textsc{T-Ref}: Then $t = \text{ref }t_1$, $\varnothing \,|\, \Sigma \vdash t_1 : T_1$, and $T = \text{Ref }T_1$. Case analysis on the last rule used in the derivation of $t \,|\, \mu \to t^\prime \,|\, \mu^\prime$.
        \begin{itemize}
        \item Case \textsc{E-Ref}: Then $t_1 \,|\, \mu \to t_1^\prime \,|\, \mu^\prime$ and $t^\prime = \text{ref }t_1^\prime$. By the induction hypothesis, there exists a $\Sigma^\prime$ such that $\Sigma^\prime \supseteq \Sigma$, $\varnothing \,|\, \Sigma^\prime \vdash t_1^\prime : T_1$, and $\varnothing \,|\, \Sigma^\prime \vdash \mu^\prime$. Then $\varnothing \,|\, \Sigma^\prime \vdash \text{ref }t_1^\prime : \text{Ref }T_1$ by \textsc{T-Ref}.
        \item Case \textsc{E-RefV}: Then $t_1$ is a value $v_1$, $t^\prime$ is a location $\ell \notin \text{dom}(\mu)$, and $\mu^\prime = \mu \cup \{ \ell \mapsto v_1 \}$. Let $\Sigma^\prime = \Sigma$. Show the following:
          \begin{itemize}
          \item $\varnothing \,|\, \Sigma \vdash \ell : \text{Ref }T_1$: Holds by \textsc{T-Loc}.
          \item $\varnothing \,|\, \Sigma \vdash \mu^\prime$:
          \end{itemize}
        \end{itemize}
      \end{itemize}

  \end{enumerate}

% PROBLEM 3
\item Freeing Refs
  \begin{enumerate}

  \item 3a

  \item 3b

  \item 3c

  \end{enumerate}

% PROBLEM 4
\item
  \begin{enumerate}

  % PROBLEM 4a
  \item Augmented Type Rules

    \begin{equation*}
      \trfrac[T-Throw]{}{
        \Gamma \vdash \mathtt{throw : T}\ |\ \{ exn \}
      }
    \end{equation*}

    Requires that $t_1$ throw:

    \begin{equation*}
      \trfrac[T-TryCatch]{
        \Gamma \vdash t_1\mathtt{ : T}\ |\ \Phi_1 \qquad
        \Gamma \vdash t_2\ \mathtt{: T}\ |\ \Phi_2 \qquad
        exn \in \Phi_1
      }{
        \Gamma \vdash \mathtt{try}\ t_1\ \mathtt{catch}\ t_2\ \mathtt{: T}\ |\
        (\Phi_1 \setminus \{ exn \}) \cup \Phi_2
      }
    \end{equation*}

    Does not require that $t_1$ throw:

    \begin{equation*}
      \trfrac[T-TryCatch]{
        \Gamma \vdash t_1\mathtt{ : T}\ |\ \Phi_1 \qquad
        \Gamma \vdash t_2\ \mathtt{: T}\ |\ \Phi_2
      }{
        \Gamma \vdash \mathtt{try}\ t_1\ \mathtt{catch}\ t_2\ \mathtt{: T}\ |\
        (\Phi_1 \setminus \{ exn \}) \cup \Phi_2
      }
    \end{equation*}


  % PROBLEM 4b
  \item Derivation
  \end{enumerate}

\end{enumerate}


    \begingroup
    \fontsize{9pt}\selectfont
    \begin{equation*}
      \trfrac[T-TryCatch]{
        \trfrac[T-App]{
          \trfrac[T-Abs]{
            \trfrac[T-Throw]{}{
              \emptyset, \mathtt{x:Bool} \vdash \mathtt{throw : Bool}\ |\ \{ exn \}
            }
          }{
            \emptyset \vdash \mathtt{(\lambda x:Bool.throw)}:\ \mathtt{Bool} \overset{\{ exn \}}{\rightarrow}\ \mathtt{Bool}\ |\ \{\}
          }
          \qquad
          \trfrac[T-True]{}{
            \emptyset \vdash \mathtt{true : Bool}\ |\ \{\}
          }
        }{
          \emptyset \vdash \mathtt{((\lambda x:Bool.throw)\ true)} : \mathtt{Bool}\ |\ \{ exn \} \cup \{\} \cup \{\}
        }
        \qquad
        \trfrac[T-False]{}{
          \emptyset \vdash \mathtt{false : Bool}\ |\ \{\}
        }
      }{
        \emptyset \vdash \mathtt{try\ ((\lambda x:Bool.throw)\ true)\ catch\ false} : \mathtt{Bool}\ |\ \{\}
      }
    \end{equation*}
    \endgroup


\end{document}

%%% End
%%%%%%%%%%%%%%%%%%%%%%%%%%%%%%%%%%%%%%%%%%%%%%%%%%%%%%%%%%%%%%%%%%%%%%%%%%%%%%%%
