\documentclass{article}
\topmargin = 0in
\oddsidemargin = 0in
\evensidemargin = \oddsidemargin
\textwidth = 6.5in
\textheight = 8in
\usepackage{times}
\usepackage{bcprules}
\usepackage{amsthm}
\usepackage{syntax}
\usepackage{trfrac}
\usepackage{mathtools}

\renewcommand{\syntleft}{\normalfont\itshape}
\renewcommand{\syntright}{}

\newcommand{\step}[2]{{\tt #1} $\longrightarrow$ {\tt #2}}
\newcommand{\eval}[2]{{\tt #1} $\Downarrow$ {\tt #2}}
\newcommand{\tc}[3]{{\tt #1} $\vdash$ {\tt #2} \ : \ {\tt #3}}
\newcommand{\tcDef}[2]{{\tt #1}\ : \ {\tt #2}}

\newcommand{\inferrule}[3]{\infrule[#1]{\mbox{#2}}{\mbox{#3}}}
\newcommand{\inferax}[2]{\infrule[#1]{\mbox{}}{\mbox{#2}}}

\newcommand{\term}[1]{{\tt t$_{#1}$}}
\newcommand{\mathType}[2]{\mathtt{#1}:\mathtt{#2}}


\title{Homework 3}

\author{Eric Marcin-Cuddy and John Bender, CS 231}

\date{November 19th, 2013}

\begin{document}

\maketitle

\begin{enumerate}
  \item \texttt{while} loop
    \begin{enumerate}
      \item Operational Semantics
        \begin{equation*}
          \trfrac[E-While]{
            t_1 \rightarrow t_1'
          }{
            \mathtt{while}\  t_1\ \mathtt{do}\ t_2 \rightarrow
            (\mathtt{while}\  t_1'\ \mathtt{do}\ t_2)\mathtt{;}\;t_2
          }
        \end{equation*}

      \item Type Rule
        \begin{equation*}
          \trfrac[T-While]{
            \Gamma \vdash t_1 : \mathtt{Bool} \qquad \Gamma \vdash t_2 : \mathtt{Unit}
          }{
            \Gamma \vdash \mathtt{while}\  t_1\ \mathtt{do}\ t_2 : \mathtt{Unit}
          }
        \end{equation*}
      \item \texttt{whileFun}
        \begin{align*}
          & \mathtt{let\ rec\ whileFun} =\\
          &\qquad \lambda x\mathtt{:Unit}.\mathtt{whileFun}\ x\ t_1\ t_2\mathtt{;}x
        \end{align*}

      \item \texttt{whileFun2}
        \begin{align*}
          & \mathtt{let\ rec\ whileFun} =\\
          &\qquad \lambda x\mathtt{:Unit}.
              \lambda t_1\mathtt{:Unit}.
                \lambda t_2\mathtt{:Unit}.t_2\mathtt{;whileFun\ unit}\ t_1\ t_2\mathtt{;unit}
        \end{align*}

    \end{enumerate}

  \item Preservation for Booleans and Integers


     \newpage

     \item Derive terms:
       \begin{enumerate}
         \item \verb|f (if x then 0 else 1)| \\

           \begin{equation*}
             \Gamma = \{ \mathType{x}{Bool}, \mathType{f}{Int \rightarrow T} \}
           \end{equation*}
           \begin{equation*}
             \trfrac[T-App]{
               \trfrac[T-Var]{
                 \mathtt{f : Int} \rightarrow \mathtt{T} \in \Gamma
               }{
                 \Gamma \vdash \mathtt{f : Int} \rightarrow \mathtt{T}
               }
               \qquad
               \trfrac[T-If]{
                 \Gamma \vdash
                 \qquad
                 \trfrac[T-Var]{
                   \mathType{x}{Bool} \in \Gamma
                 }{
                   \mathType{x}{Bool}
                 }
                 \qquad
                 \trfrac[T-Num]{}{
                   \mathType{0}{Int}
                 }
                 \qquad
                 \trfrac[T-Num]{}{
                   \mathType{1}{Int}
                 }
               }{
                 \Gamma \vdash \mathtt{(if\ x\ then\ 0\ else\ 1) : Int}
               }
             }{
               \Gamma \vdash \mathtt{f\ (if\ x\ then\ 0\ else\ 1) : T}
             }
           \end{equation*}

         \item \verb|f (if (f x) then 0 else 1)| \\

           \begin{equation*}
             \Gamma = \{ \mathType{x}{Int}, \mathType{f}{Int \rightarrow Bool} \}
           \end{equation*}
           \begin{equation*}
             \trfrac[T-App]{
               \trfrac[T-Var]{
                 \mathtt{f : Int} \rightarrow \mathtt{Bool} \in \Gamma
               }{
                 \Gamma \vdash \mathtt{f : Int} \rightarrow \mathtt{Bool}
               }
               \ \
               \trfrac[T-If]{
                 \Gamma \vdash
                 \ \
                 \trfrac[T-App]{
                   \trfrac[T-Var]{
                     \mathtt{f : Int} \rightarrow \mathtt{Bool} \in \Gamma
                   }{
                     \Gamma \vdash \mathtt{f : Int} \rightarrow \mathtt{Bool}
                   }
                   \ \
                   \trfrac[T-Var]{
                     \mathType{x}{Int} \in \Gamma
                   }{
                     \Gamma \vdash \mathType{x}{Int}
                   }
                 }{
                   \mathType{f\ x}{Bool}
                 }
                 \ \
                 \trfrac[T-Num]{}{
                   \mathType{0}{Int}
                 }
                 \ \
                 \trfrac[T-Num]{}{
                   \mathType{1}{Int}
                 }
               }{
                 \Gamma \vdash \mathtt{(if\ (f\ x)\ then\ 0\ else\ 1) : Int}
               }
             }{
               \Gamma \vdash \mathtt{f\ (if\ (f\ x)\ then\ 0\ else\ 1) : Bool}
             }
           \end{equation*}

         \item \verb|f (if (f x) then (f 0) else 1)| \\

           Won't check because the type of \verb|f| would have to be \verb|Int| $\longrightarrow$ \verb|Bool| for the \term{1} of the \verb|if| and \verb|Int| $\longrightarrow$ \verb|Int| for the \term{2} simultaneously.
       \end{enumerate}

     \item Reverse soundness
       \begin{enumerate}
         \item Reverse Progress.

           \begin{theorem}
             If $\emptyset \vdash$ \term{}$'$\verb|:T|, there exists some \term{} such that \verb|t| $\longrightarrow$ \verb|t|$'$.
           \end{theorem}

           \begin{proof}
             It is always the case that we can construct a term that steps to some \term{}$'$ by wrapping \term{}$'$ in a capture avoiding thunk and applying it. For example, $(\lambda$\verb|x:unit.|\term{}$')$ \verb|unit|, while ensuring \verb|x| is not free in \term{}$'$.
           \end{proof}

         \item Reverse Preservation.
           \begin{theorem}
             If $\emptyset \vdash$ \term{}$'$\verb|:T| and \verb|t| $\longrightarrow$ \verb|t|$'$, then $\emptyset \vdash$ \term{} \verb|:T|
           \end{theorem}

           \begin{proof}
             We proceed by case analysis on the form of \verb|t| $\longrightarrow$ \verb|t|$'$.
             \begin{enumerate}
             \item E-App1: \term{} has the form \term{1} \term{2}, \term{}$'$ has the form \term{1}$'$ \term{2}, \term{1}$'$ has type \verb|T|$_2$ $\rightarrow$ \verb|T|, and \term{2} has type \verb|T|$_2$. We proceed by the last rule in \term{}$'$\verb|:T|
               \begin{enumerate}
                 \item T-Unit: Contradiction on the form of \term{}$'$
                 \item T-Var: Contradiction on the form of \term{}$'$
                 \item T-Fun: Contradiction on the form of \term{}$'$
                 \item T-App: By the inductive hypothesis \term{1} has the type \verb|T|$_2$ $\rightarrow$ \verb|T| and \term{2} is preserved with type \verb|T|$_2$ so \term{} has type \verb|T|.
               \end{enumerate}
             \item E-App2: By a similar argument.
             \item E-AppBeta: \term{} has the form $(\lambda$\verb|x:T|$_2$\verb|.y|$)$ \verb|z| the only type rule that applies is T-Fun by which we know that \term{} must be type \verb|T|$_2$ $\rightarrow$ \verb|T|.
             \end{enumerate}
           \end{proof}

       \end{enumerate}
     \end{enumerate}
\end{document}
