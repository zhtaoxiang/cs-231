\documentclass{article}
\topmargin = 0in
\oddsidemargin = 0in
\evensidemargin = \oddsidemargin
\textwidth = 6.5in
\textheight = 8in
\usepackage{times}
\usepackage{bcprules}
\usepackage{amsthm}
\usepackage{syntax}

\renewcommand{\syntleft}{\normalfont\itshape}
\renewcommand{\syntright}{}

\newcommand{\step}[2]{{\tt #1} $\longrightarrow$ {\tt #2}}
\newcommand{\eval}[2]{{\tt #1} $\Downarrow$ {\tt #2}}
\newcommand{\tc}[3]{{\tt #1} $\vdash$ {\tt #2} \ : \ {\tt #3}}
\newcommand{\tcDef}[2]{{\tt #1}\ : \ {\tt #2}}

\newcommand{\inferrule}[3]{\infrule[#1]{\mbox{#2}}{\mbox{#3}}}
\newcommand{\inferax}[2]{\infrule[#1]{\mbox{}}{\mbox{#2}}}

\title{Homework 1}

\author{John Bender and Lorenzo Gomez, CS 231}

\date{October 8th, 2013}

\newtheorem{prog}{Progress}
\newtheorem{deter}{Determinism}

\begin{document}

\maketitle

\begin{description}
  \item[2]{Proof of Progress}
    \begin{prog}For every \verb|t|, either \verb|t| is a value or there exists a term \verb|t'| such that \verb|t| $\longrightarrow$ \verb|t'|.
    \end{prog}

    \textit{Induction Hypothesis.} For every \verb|t|$_0$ that is a subterm of \verb|t|, either \verb|t|$_0$  is a value or there exists a term \verb|t|$_0'$  such that \verb|t|$_0$  $\longrightarrow$ \verb|t|$_0'$ .

    \begin{proof}
      We proceed by case analysis on the structure of \verb|t|. \verb|true| is a value. \verb|false| is a value. For \verb|if t|$_1$ \verb|then t|$_2$ \verb|else t|$_3$ by the inductive hypothesis \verb|t|$_1$ is either a value, \verb|true| or \verb|false| and E-IfTrue or E-IfFalse apply or \verb|t|$_1$ can take a step to \verb|t|$_1'$ and E-If applies.
    \end{proof}

  \item[3]{Induction Hypothesis for Proof of Progress}
    \vspace{0.3cm}\\
    \textit{Induction Hypothesis.} For every \verb|t|$_0$ that is a subterm of \verb|t|, either \verb|t|$_0$  is a value or there exists a term \verb|t|$_0'$  such that \verb|t|$_0$  $\longrightarrow$ \verb|t|$_0'$ .

  \item[4]{Modified Semantics}
    \inferax{E-IfTrue}{\step{if true then v$_2$ else v$_3$}{v$_2$}}

    \inferax{E-IfFalse}{\step{if false then v$_2$ else v$_3$}{v$_3$}}

    \inferrule{E-IfThen}{\step{t$_2$}{t$_2'$}}
    {\step{if t$_1$ then t$_2$ else t$_3$}{if t$_1$ then t$_2'$ else t$_3$}}

    \inferrule{E-IfElse}{\step{t$_3$}{t$_3'$}}
    {\step{if t$_1$ then v$_2$ else t$_3$}{if t$_1$ then v$_2$ else t$_3'$}}

    \inferrule{E-IfGuard}{\step{t$_1$}{t$_1'$}}
    {\step{if t$_1$ then v$_2$ else v$_3$}{if t$_1'$ then v$_2$ else v$_3$}}

\newpage

  \item[5]{Proof of Determinism}
    \begin{deter}If \verb|t| $\longrightarrow$ \verb|t|$'$ and \verb|t|$'$ $\longrightarrow$ \verb|t|$''$, then \verb|t|$'$=\verb|t|$''$.
    \end{deter}

    \textit{Induction Hypothesis.} If \verb|t|$_0$ $\longrightarrow$ \verb|t|$_0'$ , \verb|t|$_0$ $\longrightarrow$ \verb|t|$_0''$, and \verb|t|$_0$ $\longrightarrow$ \verb|t|$_0'$ is a sub-derivation of \verb|t| $\longrightarrow$ \verb|t|$'$, then \verb|t|$_0'$ = \verb|t|$_0''$.

    \begin{proof}
      We proceed by case analysis on the structure of \verb|t| $\longrightarrow$ \verb|t|$'$, show that only one derivation can apply, and show that there is only one outcome for each.

      \begin{enumerate}
        \item E-IfTrue: \verb|t|$_1$ must be \verb|true|, and we address the other two cases:
          \begin{enumerate}
            \item E-IfFalse: \verb|t|$_1$ is \verb|false|, a contradiction.
            \item E-If: \verb|t|$_1$ must take a step but \verb|true| is a value.
          \end{enumerate}

          Finally, \verb|t|$_2$ is preserved by E-IfTrue so \verb|t|$'$ and \verb|t|$''$ must be \verb|t|$_2$.

        \item E-IfFalse: A similar argument holds.

        \item E-If: \verb|t|$_1$ must take a step so we address the other two cases:
          \begin{enumerate}
            \item E-IfTrue: \verb|t|$_1$ is \verb|true|, and can't take a step.
            \item E-IfFalse: A similar argument holds.
          \end{enumerate}

          By the inductive hypothesis we know for any \verb|t|$_1$ $\longrightarrow$ \verb|t|$_1'$ and \verb|t|$_1$ $\longrightarrow$ \verb|t|$_1''$, that \verb|t|$_1'$ = \verb|t|$_1''$. Since both \verb|t|$_2$ and \verb|t|$_3$ are preserved, \verb|t|$'$ and \verb|t|$''$ must be the same, \verb|if t|$_1'$ \verb|then t|$_2$ \verb|else t|$_3$.
      \end{enumerate}
    \end{proof}

  \item[6]{Induction Hypothesis for Proof of Determinism}
    \vspace{0.3cm}\\
    \textit{Induction Hypothesis.} If \verb|t|$_0$ $\longrightarrow$ \verb|t|$_0'$ , \verb|t|$_0$ $\longrightarrow$ \verb|t|$_0''$, and \verb|t|$_0$ $\longrightarrow$ \verb|t|$_0'$ is a sub-derivation of \verb|t| $\longrightarrow$ \verb|t|$'$, then \verb|t|$_0'$ = \verb|t|$_0''$.

  \item[7]{Addition of Stuck Terms to BNF}

    \begin{grammar}
      <s> :=  <b>  \lit*{+} <n> | <n> \lit*{+} <b> | <b> \lit*{+} <b>
      \alt <b>  \lit*{>} <n> | <n> \lit*{>} <b> | <b> \lit*{>} <b>
      \alt \lit*{if} <n> \lit*{then} <t> \lit*{else} <t>

      <b> := \lit*{true} | \lit*{false}

      <n> := \ldots | \lit*{-1} | \lit*{0} | \lit*{1} | \ldots

      <t> := <t> \lit*{+} <t> | <t> \lit*{>} <t> | \lit*{if} <t> \lit*{then} <t> \lit*{else} <t>
    \end{grammar}

  \item[8]{Stuck Terms, Old vs. New Semantics}
    \begin{description}
      \item[(a)] \verb|if 1 then (1 > 2) else false|
      \item[(b)] \verb|if true then false else (true > 1)|
    \end{description}

  \item[9]{Eventually Stuck Terms, Old vs. New Semantics}
    \begin{description}
      \item[(a)] None, because the set of eventually stuck terms of the new semantics for the \verb|if| form is a proper superset of the eventually stuck terms of the original semantics
      \item[(b)] \verb|if true then (1 > 0) else (true > 1)|
    \end{description}


\end{description}

\end{document}
