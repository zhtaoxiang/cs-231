\documentclass{article}
\topmargin = 0in
\oddsidemargin = 0in
\evensidemargin = \oddsidemargin
\textwidth = 6.5in
\textheight = 8in
\usepackage{times}
\usepackage{bcprules}
\usepackage{amsthm}
\usepackage{syntax}
\usepackage{trfrac}
\usepackage{mathtools}

\renewcommand{\syntleft}{\normalfont\itshape}
\renewcommand{\syntright}{}

\newcommand{\step}[2]{{\tt #1} $\longrightarrow$ {\tt #2}}
\newcommand{\eval}[2]{{\tt #1} $\Downarrow$ {\tt #2}}
\newcommand{\tc}[3]{{\tt #1} $\vdash$ {\tt #2} \ : \ {\tt #3}}
\newcommand{\tcDef}[2]{{\tt #1}\ : \ {\tt #2}}

\newcommand{\inferrule}[3]{\infrule[#1]{\mbox{#2}}{\mbox{#3}}}
\newcommand{\inferax}[2]{\infrule[#1]{\mbox{}}{\mbox{#2}}}

\newcommand{\term}[1]{{\tt t$_{#1}$}}
\newcommand{\mathType}[2]{\mathtt{#1}:\mathtt{#2}}


\title{Homework 3}

\author{John Bender and Lorenzo Gomez, CS 231}

\date{October 24th, 2013}

\newtheorem{theorem}{Theorem}


\begin{document}

\maketitle

\begin{enumerate}
  \item Progress for Booleans and Integers

    \begin{theorem}
      If \verb|t:T|, then either \verb|t| is a value or there exists some term \verb|t|$'$ such that \verb|t| $\longrightarrow$ \verb|t|$'$.
    \end{theorem}

    \begin{proof}
      We proceed by induction on the derivation of \verb|t:T| with case analysis on the last rule in the derivation.
      \begin{enumerate}
        \item T-True: \term{} is \verb|true|, a value.
        \item T-False: \term{} is \verb|false|, a value.
        \item T-Num: \term{} is \verb|n|, an integer value.
        \item T-If: \term{} is \verb|if t|$_1$ \verb|then t|$_2$ \verb|else t|$_3$, and \term{1} has type \verb|Bool|. By the inductive hypothesis \term{1} is either a value or there exists \term{1}$'$ such that \step{t$_1$}{\term{1}$'$}. We proceed by case analysis on the form of \term{1}.
          \begin{enumerate}
            \item Value: If \term{1} is a value then, by its type \verb|Bool|, we know it must be \verb|true| or \verb|false| and it can step by E-IfTrue or E-IfFalse.
            \item Steps to \term{1}$'$: Then E-If applies and \term{} can step to \verb|if t|$_1'$ \verb|then t|$_2$ \verb|else t|$_3$.
          \end{enumerate}
        \item T-Add: \term{} is \term{1} \verb|+| \term{2}. \term{1} and \term{2} have the type \verb|Int|. By the inductive hypothesis either \term{1} is a value or there exists some \term{1}$'$ such that \step{t$_1$}{\term{1}$'$}. We proceed by case analysis on the form of \term{1}.
          \begin{enumerate}
            \item Value: Either E-Plus2 or E-PlusRed may apply to \term{}. By the inductive hypothesis \term{2} is either a value, in which case E-PlusRed applies and \term{} steps to \verb|n|$_1$ $[[+]]$ \verb|n|$_2$, or \term{2} can take a step to \term{2}$'$ and E-Plus2 applies. Then \term{} steps to \verb|v|$_1$ \verb|+| \term{2}$'$.
            \item Steps to \term{1}$'$: If \term{1} can step then E-Plus1 applies and \term{} steps to \term{1}$'$ \verb|+| \term{2}.
          \end{enumerate}
        \item T-GT: By a similar argument replacing occurrences of ``Plus'' with ``GT'' and $+$ with $>$.
      \end{enumerate}
    \end{proof}

  \item Preservation for Booleans and Integers

    \begin{theorem}
      If \verb|t:T| and \verb|t| $\longrightarrow$ \verb|t|$'$, then \term{1}$'$\verb|:T|.
    \end{theorem}

    \begin{proof}
      We proceed by induction on the derivation of \verb|t:T| with case analysis on the last rule in the derivation.
      \begin{enumerate}
        \item T-True: \term{} is the value \verb|true|, so we have a contradiction with \verb|t| $\longrightarrow$ \verb|t|$'$
        \item T-False: \term{} is the value \verb|false|, so we have a contradiction with \verb|t| $\longrightarrow$ \verb|t|$'$
        \item T-Num: \term{} is an integer value,  so we have a contradiction with \verb|t| $\longrightarrow$ \verb|t|$'$
        \item T-If: Then \term{} has the form \verb|if t|$_1$ \verb|then t|$_2$ \verb|else t|$_3$, \term{1} has type \verb|Bool|, and both \term{2} and \term{3} have the type \verb|T|. We proceed by case analysis on the last rule in the derivation \verb|t| $\longrightarrow$ \verb|t|$'$.
          \begin{enumerate}
            \item E-IfTrue: Then \term{1} is \verb|true| and \term{}$'$ is \term{2}. \term{2} has type \verb|t| from T-If.
            \item E-IfFalse: Then \term{1} is \verb|false| and \term{}$'$ is \term{3}. \term{3} has type \verb|t| from T-If.
            \item E-If:  Then \term{1} $\longrightarrow$ \term{1}$'$ and \term{}$'$ is \verb|if t|$_1'$ \verb|then t|$_2$ \verb|else t|$_3$. By the inductive hypothesis, \term{1}$'$:Bool and by the application of T-If, \term{}$'$ has type \verb|T|.
          \end{enumerate}
        \item T-Plus: Then \term{} has the form \term{1} \verb|+| \term{2} and both \term{1} and \term{2} have type \verb|Int|. We proceed by case analysis on the last rule in the derivation \verb|t| $\longrightarrow$ \verb|t|$'$.
          \begin{enumerate}
            \item E-Plus1: Then \term{1} $\longrightarrow$ \term{1}$'$ and \term{2} is preserved with the type \verb|Int| from T-Plus. By the inductive hypothesis and the fact that \term{1} has the type \verb|Int| from T-Plus, \term{1}$'$ has type \verb|Int|. As a result \term{}$'$ is \term{1}$'$ \verb|+| \term{2} with type \verb|Int|.
            \item E-Plus2: Then \verb|v|$_1$ is preserved with type \verb|Int| from T-Plus and \term{2} $\longrightarrow$ \term{2}$'$. By the inductive hypothesis and the fact that \term{2} has type \verb|Int| from T-Plus we know that \term{2}$'$ has type \verb|Int|. As a result \term{}$'$ is \verb|v|$_1$ \verb|+| \term{2}$'$ with type \verb|Int|
            \item E-PlusRed: Then \verb|n|$_1$ and \verb|n|$_2$ have type \verb|Int| and the result of $[[+]]$ on two integers is again an integer \verb|n|.
          \end{enumerate}
        \item T-GT: By a similar argument replacing occurrences of ``Plus'' with ``GT'' and $+$ with $>$.
      \end{enumerate}
    \end{proof}


  \item New Rules, Progress, and Preservation.
    \begin{enumerate}
      \item Remove E-IfFalse
        \begin{enumerate}
          \item \verb|if false then 0 else 0|, breaks progress because it types with T-If but cannot step without E-IfFalse so there is no \term{}$'$ and it's not a value.
          \item It does not prevent preservation since the assumption is that the term can step.
        \end{enumerate}
      \item Add \verb|0| as type \verb|Bool|
        \begin{enumerate}
          \item \verb|if 0 then 0 else 0|, breaks progress because it breaks the canonical forms lemma which is required to show that this term can take a step with E-IfTrue or E-IfFalse.
          \item Doesn't invalidate preservation because the term can't step so it's a contradiction.
        \end{enumerate}

      \item Add E-If2. Neither. Ambiguity provides for more ways to step and the rule preserves the type of an \verb|if|.

      \item Addition of booleans.
        \begin{enumerate}
          \item Doesn't invalidate progress because the new evaluation rules provide for progress and they can be typed by the new type rule.
          \item Breaks preservation. \verb|false + false| which steps to \verb|false| under the new rule but doesn't type as \verb|Int|.
        \end{enumerate}

      \item \verb|Int| guard.
        \begin{enumerate}
          \item \verb|if 1 then 0 else 0| type checks but can't step and isn't a value.
          \item Doesn't affect preservation because the assumption that the term can step is contradicted.
        \end{enumerate}
     \end{enumerate}

     \newpage

     \item Derive terms:
       \begin{enumerate}
         \item \verb|f (if x then 0 else 1)| \\

           \begin{equation*}
             \Gamma = \{ \mathType{x}{Bool}, \mathType{f}{Int \rightarrow T} \}
           \end{equation*}
           \begin{equation*}
             \trfrac[T-App]{
               \trfrac[T-Var]{
                 \mathtt{f : Int} \rightarrow \mathtt{T} \in \Gamma
               }{
                 \Gamma \vdash \mathtt{f : Int} \rightarrow \mathtt{T}
               }
               \qquad
               \trfrac[T-If]{
                 \Gamma \vdash
                 \qquad
                 \trfrac[T-Var]{
                   \mathType{x}{Bool} \in \Gamma
                 }{
                   \mathType{x}{Bool}
                 }
                 \qquad
                 \trfrac[T-Num]{}{
                   \mathType{0}{Int}
                 }
                 \qquad
                 \trfrac[T-Num]{}{
                   \mathType{1}{Int}
                 }
               }{
                 \Gamma \vdash \mathtt{(if\ x\ then\ 0\ else\ 1) : Int}
               }
             }{
               \Gamma \vdash \mathtt{f\ (if\ x\ then\ 0\ else\ 1) : T}
             }
           \end{equation*}

         \item \verb|f (if (f x) then 0 else 1)| \\

           \begin{equation*}
             \Gamma = \{ \mathType{x}{Int}, \mathType{f}{Int \rightarrow Bool} \}
           \end{equation*}
           \begin{equation*}
             \trfrac[T-App]{
               \trfrac[T-Var]{
                 \mathtt{f : Int} \rightarrow \mathtt{Bool} \in \Gamma
               }{
                 \Gamma \vdash \mathtt{f : Int} \rightarrow \mathtt{Bool}
               }
               \ \
               \trfrac[T-If]{
                 \Gamma \vdash
                 \ \
                 \trfrac[T-App]{
                   \trfrac[T-Var]{
                     \mathtt{f : Int} \rightarrow \mathtt{Bool} \in \Gamma
                   }{
                     \Gamma \vdash \mathtt{f : Int} \rightarrow \mathtt{Bool}
                   }
                   \ \
                   \trfrac[T-Var]{
                     \mathType{x}{Int} \in \Gamma
                   }{
                     \Gamma \vdash \mathType{x}{Int}
                   }
                 }{
                   \mathType{f\ x}{Bool}
                 }
                 \ \
                 \trfrac[T-Num]{}{
                   \mathType{0}{Int}
                 }
                 \ \
                 \trfrac[T-Num]{}{
                   \mathType{1}{Int}
                 }
               }{
                 \Gamma \vdash \mathtt{(if\ (f\ x)\ then\ 0\ else\ 1) : Int}
               }
             }{
               \Gamma \vdash \mathtt{f\ (if\ (f\ x)\ then\ 0\ else\ 1) : Bool}
             }
           \end{equation*}

         \item \verb|f (if (f x) then (f 0) else 1)| \\

           Won't check because the type of \verb|f| would have to be \verb|Int| $\longrightarrow$ \verb|Bool| for the \term{1} of the \verb|if| and \verb|Int| $\longrightarrow$ \verb|Int| for the \term{2} simultaneously.
       \end{enumerate}

     \item Reverse soundness
       \begin{enumerate}
         \item Reverse Progress.

           \begin{theorem}
             If $\emptyset \vdash$ \term{}$'$\verb|:T|, there exists some \term{} such that \verb|t| $\longrightarrow$ \verb|t|$'$.
           \end{theorem}

           \begin{proof}
             It is always the case that we can construct a term that steps to some \term{}$'$ by wrapping \term{}$'$ in a capture avoiding thunk and applying it. For example, $(\lambda$\verb|x:unit.|\term{}$')$ \verb|unit|, while ensuring \verb|x| is not free in \term{}$'$.
           \end{proof}

         \item Reverse Preservation.
           \begin{theorem}
             If $\emptyset \vdash$ \term{}$'$\verb|:T| and \verb|t| $\longrightarrow$ \verb|t|$'$, then $\emptyset \vdash$ \term{} \verb|:T|
           \end{theorem}

           \begin{proof}
             We proceed by case analysis on the form of \verb|t| $\longrightarrow$ \verb|t|$'$.
             \begin{enumerate}
             \item E-App1: \term{} has the form \term{1} \term{2}, \term{}$'$ has the form \term{1}$'$ \term{2}, \term{1}$'$ has type \verb|T|$_2$ $\rightarrow$ \verb|T|, and \term{2} has type \verb|T|$_2$. We proceed by the last rule in \term{}$'$\verb|:T|
               \begin{enumerate}
                 \item T-Unit: Contradiction on the form of \term{}$'$
                 \item T-Var: Contradiction on the form of \term{}$'$
                 \item T-Fun: Contradiction on the form of \term{}$'$
                 \item T-App: By the inductive hypothesis \term{1} has the type \verb|T|$_2$ $\rightarrow$ \verb|T| and \term{2} is preserved with type \verb|T|$_2$ so \term{} has type \verb|T|.
               \end{enumerate}
             \item E-App2: By a similar argument.
             \item E-AppBeta: \term{} has the form $(\lambda$\verb|x:T|$_2$\verb|.y|$)$ \verb|z| the only type rule that applies is T-Fun by which we know that \term{} must be type \verb|T|$_2$ $\rightarrow$ \verb|T|.
             \end{enumerate}
           \end{proof}

       \end{enumerate}
     \end{enumerate}
\end{document}
